\chapter{引言}\label{chap:introduction}

引力波是爱因斯坦广义相对论最重要的理论预言之一。自从2015年9月14日首次直接探测到了一对双黑洞并合产生的引力波信号(GW150914 \cite{Abbott:2016blz})以来,激光干涉引力波天文台 \cite{TheLIGOScientific:2014jea}(Laser Interferometer Gravitational-Wave Observatory,简称LIGO) 和室女座干涉仪 \cite{TheVirgo:2014hva}(Virgo interferometer,简称Virgo)为人类探测宇宙打开了一扇新的窗口,开启了引力波天文学的新时代。双黑洞并合事件的观测使我们得以检验强场区域的引力性质 \cite{TheLIGOScientific:2016src,LIGOScientific:2019fpa},并估算双黑洞的并合率以及黑洞的群体特征(比如质量和自旋的分布) \cite{LIGOScientific:2018jsj,Abbott:2020gyp}。其后,在2017年8月17日LIGO-Virgo首次探测到了一对双中子星并合产生的引力波信号(GW170817 \cite{TheLIGOScientific:2017qsa}),同时不同电磁波段的望远镜也观测到了该中子星并合产生的的电磁对应体 \cite{Monitor:2017mdv,GBM:2017lvd},标志着多信使引力波天文学新时代的来临。根据最新LIGO-Virgo科学组织发布的引力波瞬变目录2(Gravitational-Wave Transient Catalog 2,简称GWTC-2) \cite{Abbott:2020niy},目前一共探测到了50个致密双星并合的事例,其中大多数都是双黑洞并合的事例。

在\lvc 探测到第一个双黑洞并合事件(GW150914\cite{Abbott:2016blz})后,人们就试图理解引力波探测到的黑洞是如何形成的,以及黑洞是如何成对的。事实上,关于\lvc 探测到的双黑洞的成因,目前还存在争议。我们知道,通过电磁波手段(即X射线)已探测到的银河系内的双黑洞的质量大概在$5\sim 15\Msun$ \cite{Remillard:2006fc}。而GW150914事件的双黑洞质量分别为$36^{+5}_{-4}\Msun$和$29^{+4}_{-4}\Msun$ \cite{Abbott:2016blz},要远大于X射线探测到黑洞的质量。所以,有猜想认为,有别于X射线探测到的天体物理黑洞,引力波探测到的黑洞可能形成于其它机制,例如原初黑洞 \cite{Bird:2016dcv,Sasaki:2016jop,Chen:2018czv,Clesse:2017bsw}。原初黑洞是在宇宙早期由于原初密度扰动的引力塌缩而形成的黑洞\cite{Hawking:1971ei,Carr:1974nx,Khlopov:2008qy,Sasaki:2018dmp}。原初黑洞不仅可以解释\lvc 探测到的黑洞,而且是暗物质的候选者之一,可能构成部分或全部的暗物质。


为了和\lvc 探测到的双黑洞的事件率做比较,我们需要知道原初双黑洞的并合率分布,进而用引力波探测数据来限制原初黑洞的模型参数。利用黑洞质量分布的信息以及理论模型推演出来的并合率,我们将有可能回答引力波探测到的黑洞的起源以及这些黑洞是如何演化的。在\lvc 已公开的几十个双黑洞事例中,黑洞的质量并不是一样的,而是有分布的,并且质量处于$5\sim 91\Msun$之间\cite{Abbott:2020niy}。然而,在过去估算原初双黑洞并合率的研究中,人们通常假定所有原初黑洞的质量都是一样的,即假定原初黑洞的质量谱是单色的\cite{Sasaki:2016jop,Nakamura:1997sm,Ali-Haimoud:2017rtz,Bird:2016dcv,Nishikawa:2017chy}。最近,文献\cite{Raidal:2017mfl}和文献\cite{Kocsis:2017yty}都计算了有质量分布的原初黑洞的并合率。但他们的计算都有可以改进的地方。例如,文献\cite{Raidal:2017mfl}只考虑了距离原初双黑洞系统最近的第三个黑洞对双黑洞系统的潮汐作用,而忽略了其他黑洞对双黑洞系统的相互作用;而文献\cite{Kocsis:2017yty}只考虑了原初黑洞的质量谱是平的分布,而且质量谱的宽度很窄。在考虑所有原初黑洞以及线性物质密度扰动对原初双黑洞系统产生的力矩的情况下,如何计算具有一般质量谱的原初双黑洞的并合率分布,这是本论文要研究的第一个内容。

由于地基引力波探测器探测能力的限制,\lvc 引力波探测器目前只能探测到红移$z<1$以内的双黑洞并合产生的引力波事件\citep{TheLIGOScientific:2016htt,Aasi:2013wya}。除了被\lvc 探测到的双黑洞外,宇宙中还有许许多多无法被\lvc 探测到的双黑洞或其他星体并合的事件。这些致密天体并合的过程中产生的引力波会相互叠加形成随机引力波背景\citep{Christensen:1992wi}。引力波背景是引力波探测的重要波源之一,目前还未被探测到。假设\lvc 探测到的所有的黑洞都来自天体演化形成的\citep{Belczynski:2010tb,Miller:2016krr,TheLIGOScientific:2016htt,Belczynski:2016obo,Stevenson:2017tfq},文献\cite{TheLIGOScientific:2016wyq,TheLIGOScientific:2016dpb}计算了来自双黑洞产生的随机引力波背景。这些研究表明在最乐观的估计下,天体物理双黑洞产生的随机引力波背景可能在\lvc 达到其最终设计灵敏度之前就被探测到了。假设\lvc 探测到的双黑洞是原初双黑洞,如何计算原初双黑洞产生的随机引力波背景以及估算这一背景对未来的空间引力波探测器比如激光干涉空间天线(Laser Interferometer Space Antenna,简称LISA) \citep{Audley:2017drz}的影响,这是本论文要研究的第二个内容。


除了原初黑洞模型,天体物理黑洞模型也可能解释\lvc 探测到的双黑洞。天体物理双黑洞的形成和并合是由演化环境主导的。在文献中,天体物理双黑洞模型主要有三种机制。第一种是\textit{动态形成}机制,即大质量恒星的演化形成黑洞,而黑洞被分离到星团核心,最后配对形成双黑洞系统\cite{Rodriguez:2015oxa,Rodriguez:2016kxx,Park:2017zgj}。第二种是\textit{经典孤立的双星演化}机制,即双黑洞是通过质量转移或公共包层抛射(common envelope ejection)而形成的\cite{Belczynski:2014iua,Belczynski:2016obo,Woosley:2016nnw,Rodriguez:2018rmd,Choksi:2018jnq}。第三种是\textit{化学均匀演化}机制,即由于氦气在整个包络体中的混合\cite{2010AIPC.1314..291D,deMink:2016vkw},使得恒星几乎在化学物质均匀的环境种演化形成黑洞。双黑洞的群体属性,如自旋\cite{Farr:2017uvj,Tiwari:2018qch,Ng:2018neg,Stevenson:2017dlk,Bogomazov:2018prw,Lopez:2018nkj,Sedda:2018nxm,Farr:2017gtv}、红移\cite{Fishbach:2018edt,Emami:2018taj,Bai:2018shq}和偏心率分布\cite{Samsing:2013kua,Samsing:2017xmd,Samsing:2017jnz,Lower:2018seu}等性质有可能区分不同机制的天体物理双黑洞模型。在未来,随着引力波探测器的更新换代,我们将探测到越来越多的引力波事件。第三代地基引力波探测器,如爱因斯坦望远镜\cite{Punturo:2010zz}和宇宙勘探者\cite{Evans:2016mbw}有望每年探测到$\Od(10^5)$\cite{Regimbau:2016ike,Vitale:2018yhm}个双黑洞并合事件。如何通过大量的双黑洞并合事件,并利用探测到的双黑洞数目随红移演化的信息来区分这些黑洞到底是天体物理黑洞还是原初黑洞,这是本论文要研究的第三个内容。

原初黑洞是由宇宙早期的标量扰动的增强而形成的\cite{Hawking:1971ei,Carr:1974nx}。原初黑洞形成的过程将不可避免地伴随着由标量诱导的次生引力波\cite{Matarrese:1992rp,Matarrese:1993zf,Matarrese:1997ay,Noh:2004bc,Carbone:2004iv,Nakamura:2004rm,Ananda:2006af}。
这些标量诱导引力波是由辐射主导时期的标量扰动驱动的,会留下现在揖让可检测的信号,所以通过标量诱导引力波也可以间接探测原初黑洞暗物质\cite{Saito:2008jc,Bugaev:2010bb,Sasaki:2018dmp,Inomata:2018epa,Baumann:2007zm,Clesse:2018ogk,Nakama:2016enz,Saito:2009jt,Bugaev:2009zh,Assadullahi:2009jc}。由于原初黑洞是由曲率扰动概率密度的尾部形成的,所以形成单个原初黑洞的概率对曲率扰动功率谱的振幅相当敏感\cite{Young:2014ana}。因此,原初黑洞占冷暗物质的丰度对标量诱导引力波的振幅极为敏感。如果探测到标量诱导引力波将为原初黑洞的存在提供证据;而如果没有探测到标量诱导引力波将对原初黑洞的丰度给出限制。目前地基引力波探测器的可探测频率为$10\sim10^4$ Hz\cite{Martynov:2016fzi}。作为补充工具,稳定的毫秒脉冲星是天然的星系级引力波探测器。脉冲星可探测纳赫兹频段的引力波,为探索宇宙打开了一扇新的窗口。由引力引起的时空扰动会影响脉冲星的射电脉冲达到地球的时间,从而通过监测脉冲到达时间(time of arrival,简称TOA)的变化可以探测引力波。一颗脉冲星和地球构成的系统相当于一个星系级别的引力波探测器。可以把很多颗脉冲星和地球构成的系统看成一个大的引力波探测器网络,即脉冲星计时阵列(pulsar timing array,简称PTA)\cite{1978SvA....22...36S,Detweiler:1979wn,1990ApJ...361..300F},进而提高探测能力。目前世界上有三个大的脉冲星计时阵列合作组织,分别是北美纳赫兹引力波天文台(North American Nanohertz Observatory for Gravitational Waves,简称NANOGrav)\cite{McLaughlin:2013ira}、澳大利亚的帕克斯脉冲星计时阵列(Parkes Pulsar Timing Array,简称PPTA)\cite{Manchester:2012za}和欧洲脉冲星计时阵列(European Pulsar Timing Array,简称EPTA)\cite{Assadullahi:2009jc}。这些脉冲星计时阵列组织结合形成国际脉冲星计时阵列(International Pulsar Timing Array,简称IPTA)\cite{2010CQGra..27h4013H}。目前各个脉冲星计时阵列已经积累了对数十颗脉冲星长达十几年的观测数据。虽然脉冲星计时阵列还没有探测到引力波,但是脉冲星计时阵列的数据可以对各种物理过程(比如宇宙弦\cite{Lentati:2015qwp,Arzoumanian:2018saf,Yonemaru:2020bmr}、单个超大质量双黑洞产生的连续引力波\cite{Zhu:2014rta,Babak:2015lua,Aggarwal:2018mgp}、引力波的记忆效应\cite{Wang:2014zls,Aggarwal:2019ypr}以及幂律谱的随机引力波背景\cite{Lentati:2015qwp,Shannon:2015ect,Arzoumanian:2018saf})进行限制。虽然各个引力波组没有探测到以上提到的各种引力波源,但是由于标量诱导引力波背景的能量密度谱和其他引力波源的能量密度谱具有显著区别\cite{Yuan:2019wwo},所以并不能排除引力波计时阵列数据中存在标量诱导引力波的可能性。在脉冲星计时阵列的数据中搜索标量诱导引力波,进而间接探测原初黑洞暗物质,这是本论文要研究的第四个内容。

本文将研究通过引力波来探测原初黑洞,具体研究上述提到的四个方面的内容。在接下来的第二章里,我们将回顾引力波的基础知识,包括引力波的传播方程的推导、引力波的能动张量的定义、标量诱导引力波的计算以及脉冲星计时阵列探测随机引力波背景的原理。

在第三章里,我们考虑所有其他原初黑洞以及线性密度扰动产生的力矩对原初双黑洞演化的影响,然后计算了具有一般质量分布情况下的原初双黑洞的并合率分布。利用我们得到的并合率分布以及\lvc 探测到的双黑洞并合的引力波数据,我们对原初黑洞占冷暗物质的丰度给出限制。

在第四章里,我们计算了双黑洞和双中子星并合产生的随机引力波背景。我们考虑了两种不同的双黑洞形成机制,分别是天体物理双黑洞和原初双黑洞机制。并分析这一引力波背景能否被未来的空间引力波探测器(比如LISA)观测到。如果这一引力波背景没能从LISA探测器中扣除掉,将会构成LISA的额外噪音。我们进而分析了由引力波背景构成的噪音对LISA的探测能力的影响。

在第五章里,我们探讨了通过下一代地基引力波探测器,比如爱因斯坦望远镜和宇宙勘探者,来区分原初黑洞和天体物理黑洞的可能性。通过定向搜寻亚太阳质量的双黑洞系统,我们估算了原初黑洞占暗物质丰度的可探测上限。另外,我们预测了爱因斯坦望远镜和宇宙勘探者能够探测到的双黑洞事件数目随红移的分布,从而来区分原初黑洞和天体物理黑洞模型。

在第六章里,我们首次在NANOGrav 11年的脉冲星数据中搜索伴随原初黑洞形成而产生的标量诱导引力波信号。并对引力波的振幅和原初黑洞占冷暗物质的丰度给出限制。

在第七章里,我们对本文的内容做总结。
