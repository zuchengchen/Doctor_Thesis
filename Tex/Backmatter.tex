%---------------------------------------------------------------------------%
%->> Backmatter
%---------------------------------------------------------------------------%
\chapter[致谢]{致\quad 谢}\chaptermark{致\quad 谢}% syntax:
转瞬之间,我已在理论所学习了四年。临别之际,我想对所有关心和帮助过我的人表示最衷心的感谢!

我要真诚感谢我的导师黄庆国研究员对我的悉心栽培!黄老师在科研和生活方面都给予我极大的帮助。在科研方面,黄老师对物理学的前沿有敏锐的洞察力,从入学的时候就建议我从事引力波和原初黑洞这一新兴领域的研究工作。黄老师具有很强的物理直觉,我时常能从他那里获得关于物理现象的新见解。在我科研工作遇到问题的时候,黄老师总能一针见血地指明问题的关键,使得我们能够专心攻克难点,取得进展。黄老师在科研工作上具有极大的激情,他似乎无时无刻不在思考物理问题,并总能提出很多奇思妙想并乐意与我们分享。在科研遇到困难的时候,我总能随时通过各种通信工具向他求助和探讨。记得有多次为了尽快完成手上的科研工作,黄老师不惜牺牲休息的时间,与我们通过微信或小鱼易连讨论至深夜甚至凌晨。这种激情一直激励着我不断努力工作。在黄老师的指导下做研究是愉快和高效的。在生活方面,黄老师给了我诸多关照。在我博士入学之前的半年,黄老师就提前资助我来理论所访问。读博后,黄老师还多次资助我去国内外参加各种学术交流活动,并经常给我推荐就业信息。在读博期间,我还经常向黄老师请假回家与家人团聚,黄老师每次都通情达理地批准了。我再次向黄老师表示最诚挚的谢意!

我要感谢我的合作者们。他们是黄庆国老师、Vivien Raymond老师、戚虹老师和林文斌老师,以及袁晨师弟、黄帆师弟、李君师姐、罗华美学妹和吴玉梅师妹。特别是袁晨师弟,他思维活跃、物理图像清晰,我们的合作是卓有成效的,已经完成了5篇文章。


我要感谢台湾师范大学的林丰利老师以及广州大学的张靖仪老师,他们曾分别邀请和资助我去访问。此外,我还要感谢理论所提供的平台,让我有机会申请到国科大的资助,从而能够在英国卡迪夫大学进行为期一年的访问学习。为此,我要感谢黄庆国老师、刘润球老师和戚虹老师在我申请访学过程中提供的无私帮助。我要感谢在卡迪夫大学访学时的导师戚虹老师和Vivien Raymond老师,他们总能在百忙之中抽空听取我的科研进展并讨论和指导下一步工作。我还要感谢Edward Fauchon-Jones、Sebastian Khan和Jonathan E. Thompson,他们在我访学期间为我解答了关于中子星黑洞模板的各种问题。我要感谢Charlie Hoy、Virginia d'Emilio和Cecilio García-Quirós,他们曾帮我解答关于\texttt{Bilby}和\texttt{PESummary}软件包的各种问题。

我要感谢组内的同学们。王赛师兄、皮石师兄、程程师姐、张克超师兄、邢宇航师兄、王科师兄、陈璐师姐、李君师姐、桑语师兄、张雪师姐、方芸、黄帆、袁晨、吴玉梅、郭忠凯、孟德双、国荣祯和韩雨轩在平时的学习生活中,为我提供了不少帮助。特别是组里的秘书陈璐、黄帆和吴玉梅,感谢他们的辛勤付出。

我要感谢国内的室友戴卫明学长、柳浪和贾乙丁提供了良好的宿舍环境,使我得以安心学习和科研。我要感谢在英国访学期间的室友杜慕皓、王凯和孙佳时常分享他们的美食,以及在我心情低落的时候给予我诸般安慰;怀念在疫情期间一起斗地主的欢乐时光。我要感谢同办公室的曲峰和何淼,我们经常在去就餐的路上讨论各种学习和生活上的事情。

我要感谢在理论所行政岗位上默默付出的王丽老师、孙亚宁老师、郭舒婷老师和石平老师。特别是石平老师曾多次主动提醒和联系我完成一些行政上的事务。

我要感谢我的硕士导师韦浩教授。韦老师不仅手把手地引导我进入科研领域,还在我读博期间时常关心我的动态。

我要感谢我的家人们对我的理解、包容和支持。感谢父母几十年来的养育之恩!感谢岳父母对我妻子和孩子的照顾!感谢妻子对我的支持!

我要感谢上帝在我人生每个阶段的带领。我要感谢教会的众弟兄姊妹们,让我不管是身处国内还是异国他乡都能感受到上帝的爱。

谨以此文献给我的儿子陈曦。

\vspace{1cm}
\hspace{10cm} 2021年4月27日

\chapter{作者简历及攻读学位期间发表的学术论文与研究成果}

\section*{作者简历:}

2008年9月至2012年6月,在福州大学紫金矿业学院获学士学位。

2012年9月至2015年3月,在北京理工大学物理学院获硕士学位。

2015年9月至2016年10月,在湖南文理学院任教师。

2019年10月至2020年10月,在英国卡迪夫大学物理与天文系访学(加入LIGO)。

2017年9月至2021年6月,在中国科学院理论物理研究所攻读博士学位。

\section*{短作者文章:}
\begin{enumerate}
    \item \href{https://arxiv.org/abs/2101.06869}{Non-tensorial Gravitational Wave Background in NANOGrav 12.5-Year Data Set}\\ \href{https://arxiv.org/abs/2101.06869}{arXiv:2101.06869 [astro-ph.CO]}\\
    \textbf{Zu-Cheng Chen}, Chen Yuan, Qing-Guo Huang
    
    \item \href{https://journals.aps.org/prd/abstract/10.1103/PhysRevD.101.063018}{Scalar induced gravitational waves in different gauges}\\
    Physical Review D, 2020, 101(6): 063018, \href{https://arxiv.org/abs/1912.00885}{arXiv:1912.00885 [astro-ph.CO]}\\
    Chen Yuan, \textbf{Zu-Cheng Chen}, Qing-Guo Huang
    
    \item \href{https://link.springer.com/article/10.1007\%2Fs11467-019-0936-x}{Extraction of gravitational wave signals with optimized convolutional neural network}\\
    Frontiers of Physics, 2020, 15(1): 14601\\
    Hua-Mei Luo, Wenbin Lin, \textbf{Zu-Cheng Chen}, Qing-Guo Huang    
    
    \item \href{https://journals.aps.org/prl/abstract/10.1103/PhysRevLett.124.251101}{Pulsar Timing Array Constraints on Primordial Black Holes with NANOGrav 11-Year Dataset}\\
    Physical Review Letters, 2020, 124(25): 251101, \href{https://arxiv.org/abs/1910.12239}{arXiv:1910.12239 [astro-ph.CO]}\\    
    \textbf{Zu-Cheng Chen}, Chen Yuan, Qing-Guo Huang    
    
    \item \href{https://journals.aps.org/prd/abstract/10.1103/PhysRevD.101.043019}{Log-dependent slope of scalar induced gravitational waves in the infrared regions}\\
    Physical Review D, 2020, 101(4): 043019, \href{https://arxiv.org/abs/1910.09099}{arXiv:1910.09099 [astro-ph.CO]}\\ 
    Chen Yuan, \textbf{Zu-Cheng Chen}, Qing-Guo Huang
    
    \item \href{https://link.springer.com/article/10.1007/s11433-019-9605-5}{Measuring the tilt of primordial gravitational-wave power spectrum from observations}\\
    Science China (Physics, Mechanics \& Astronomy), 2019 (11): 16, \href{https://arxiv.org/abs/1907.09794}{arXiv:1907.09794 [astro-ph.CO]}\\
    Jun Li, \textbf{Zu-Cheng Chen}, Qing-Guo Huang
    
    \item \href{https://journals.aps.org/prd/abstract/10.1103/PhysRevD.100.081301}{Probing primordial-black-hole dark matter with scalar induced gravitational waves}\\
    Physical Review D, 2019, 100(8): 081301, \href{https://arxiv.org/abs/1906.11549}{arXiv:1906.11549 [astro-ph.CO]}\\
    Chen Yuan, \textbf{Zu-Cheng Chen}, Qing-Guo Huang
    
    \item \href{https://iopscience.iop.org/article/10.1088/1475-7516/2020/08/039}{Distinguishing Primordial Black Holes from Astrophysical Black Holes by Einstein Telescope and Cosmic Explorer}\\
    Journal of Cosmology and Astroparticle Physics, 2020, 2020(08): 039, \href{https://arxiv.org/abs/1904.02396}{arXiv:1904.02396 [astro-ph.CO]}\\
    \textbf{Zu-Cheng Chen},Qing-Guo Huang
    
    \item \href{https://iopscience.iop.org/article/10.3847/1538-4357/aaf581}{Stochastic Gravitational-Wave Background from Binary Black Holes and Binary Neutron Stars and Implications for LISA}\\
    The Astrophysical Journal, 2019, 871(1): 97, \href{https://arxiv.org/abs/1809.10360}{arXiv:1809.10360 [gr-qc]}\\
    \textbf{Zu-Cheng Chen}, Fan Huang, Qing-Guo Huang
    
    \item \href{https://iopscience.iop.org/article/10.3847/1538-4357/aad6e2}{Merger Rate Distribution of Primordial-Black-Hole Binaries}\\
    The Astrophysical Journal, 2018, 864(1): 61, \href{https://arxiv.org/abs/1801.10327}{arXiv:1801.10327 [astro-ph.CO]}\\
    \textbf{Zu-Cheng Chen},Qing-Guo Huang
\end{enumerate} 

\section*{长作者文章:}
\begin{enumerate}
    
    \item \href{https://arxiv.org/abs/2103.08520}{Search for anisotropic gravitational-wave backgrounds using data from Advanced LIGO's and Advanced Virgo's first three observing runs}\\ \href{https://arxiv.org/abs/2103.08520}{arXiv:2103.08520 [gr-qc]}\\
    LIGO Scientific and Virgo and KAGRA Collaborations
    
    \item \href{https://arxiv.org/abs/2101.12248}{Constraints on cosmic strings using data from the third Advanced LIGO-Virgo observing run}\\ \href{https://arxiv.org/abs/2101.12248}{arXiv:2101.12248 [gr-qc]}\\
    LIGO Scientific and Virgo and KAGRA Collaborations

    \item \href{https://arxiv.org/abs/2101.12130}{Upper Limits on the Isotropic Gravitational-Wave Background from Advanced LIGO's and Advanced Virgo's Third Observing Run}\\ \href{https://arxiv.org/abs/2101.12130}{arXiv:2101.12130 [gr-qc]}\\
    LIGO Scientific and Virgo and KAGRA Collaborations    
    
    \item \href{https://arxiv.org/abs/2012.12926}{Diving below the spin-down limit: Constraints on gravitational waves from the energetic young pulsar PSR J0537-6910}\\ \href{https://arxiv.org/abs/2012.12926}{arXiv:2012.12926 [astro-ph.HE]}\\
    LIGO Scientific and Virgo and KAGRA Collaborations      
    
    \item \href{https://journals.aps.org/prd/abstract/10.1103/PhysRevD.103.064017}{All-sky search in early O3 LIGO data for continuous gravitational-wave signals from unknown neutron stars in binary systems}\\     
    Physical Review D, 2021, 103(6): 064017, \href{https://arxiv.org/abs/2012.12128}{arXiv:2012.12128 [gr-qc]}\\
    LIGO Scientific and Virgo and KAGRA Collaborations    
\end{enumerate}


\section*{获奖情况:}
\begin{itemize}
    \item 2020-2021学年中国科学院大学三好学生标兵
    
    \item 2020年理论物理研究所曙光优秀奖
    
    \item 2019年博士研究生国家奖学金
    
    \item 2018-2019学年中国科学院大学三好学生
    
    \item 2018年理论物理研究所曙光特别奖
    
    \item 2012年硕士研究生国家奖学金
    
\end{itemize}



%\chapter[目录]{标题}\chaptermark{页眉}
\thispagestyle{noheaderstyle}% 如果需要移除当前页的页眉
%\pagestyle{noheaderstyle}% 如果需要移除整章的页眉



\cleardoublepage[plain]% 让文档总是结束于偶数页,可根据需要设定页眉页脚样式,如 [noheaderstyle]
%---------------------------------------------------------------------------%
