\chapter{总结}

本文探索通过引力波来探测原初黑洞。首先,考虑所有其他原初黑洞以及线性密度扰动产生的力矩对原初双黑洞演化的影响,我们计算了具有一般质量分布情况下的原初双黑洞的并合率分布。利用我们得到的并合率分布以及\lvc 探测到的双黑洞并合的引力波数据,我们对原初黑洞占冷暗物质的丰度给出的限制为$10^{-3} \lesssim \fpbh \lesssim 10^{-2}$,证实了绝大多数的冷暗物质不是由恒星级质量的原初黑洞构成的。

其次,我们计算了双黑洞和双中子星并合产生的随机引力波背景。我们考虑了两种不同的双黑洞形成机制,分别是天体物理双黑洞和原初双黑洞机制。并分析这一引力波背景能否被未来的引力波探测器(比如LISA)观测到。我们发现双黑洞和双中子星产生的随机引力波背景可以被未来的空间引力波探测器LISA探测到。如果这一引力波背景没能从LISA探测器中扣除掉,将会构成LISA的额外噪音,从而降低LISA的探测能力。

然后,我们探讨了通过下一代地面引力波探测器,比如ET和CE,来区分原初黑洞和天体物理黑洞的可能性。通过定向搜寻亚太阳质量的双黑洞系统,我们估算了原初黑洞占暗物质丰度的可探测下限。另外,我们预测了ET和CE能够探测到的双黑洞事件数目随红移的分布,从而来区分原初黑洞和天体物理黑洞模型。

最后,我们首次在NANOGrav 11年的脉冲星计时阵列数据集中搜索伴随原初黑洞形成而产生的标量诱导引力波信号。由于没有发现统计意义上显著的引力波信号,我们对质量在$[2 \times 10^{-3}, 7\times 10^{-1}]$太阳质量区间的原初黑洞占冷暗物质的丰度给出了迄今为止最严格的限制。

